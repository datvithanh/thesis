\documentclass[a4paper, 13pt, oneside]{report}
\usepackage[utf8]{inputenc}
\usepackage{amsfonts}
\usepackage[numbers,sort&compress]{natbib}
\usepackage{comment}
\usepackage[export]{adjustbox}
\usepackage[utf8]{vietnam}
\usepackage{amssymb}
\usepackage{url}
\usepackage{fancybox}
\usepackage{multirow}
\usepackage{multicol}
\usepackage{graphicx}
\usepackage{subfiles}
\usepackage[left=3.50cm, right=2.00cm, top=2.00cm, bottom=2.00cm]{geometry}
\usepackage{fancyhdr}
\usepackage{hyperref}
\usepackage{changepage}
\usepackage{framed}
\usepackage{multirow}
\usepackage{diagbox}
\usepackage{amsmath}
\usepackage{bm}
\usepackage{tabu}
\usepackage{booktabs}
\usepackage{listings}
\usepackage{placeins}
\usepackage{multirow}
\usepackage{setspace}
% \usepackage[backend=biber, sorting=none]{biblatex}
\usepackage{listings}
\usepackage[nottoc]{tocbibind}
\usepackage[table,xcdraw]{xcolor}
\usepackage{float}
\floatstyle{plaintop}
\restylefloat{table}
\usepackage{caption}
\usepackage{subcaption}
\usepackage[english]{babel}
\usepackage{amssymb}
\usepackage{pifont}
% \usepackage{subfig}

\usepackage{bm}
\usepackage[acronym, nonumberlist, shortcuts, toc]{glossaries}
\newcommand{\cmark}{\ding{51}}
\newcommand{\xmark}{\ding{55}}

\usepackage[nowatermark]{fixmetodonotes}

\usepackage[linesnumbered,ruled,vlined]{algorithm2e}
\usepackage{algorithmic}
\SetAlFnt{\footnotesize}
\SetKw{KwDownTo}{downto}
\SetKw{KwTrue}{true}
\SetKw{KwFalse}{false}
\SetKwInOut{Input}{Input}
\SetKwInOut{Output}{Output}
\SetKw{KwAnd}{and}
\makeatletter
\newcommand{\nosemic}{\renewcommand{\@endalgocfline}{\relax}}
\newcommand{\dosemic}{\renewcommand{\@endalgocfline}{\algocf@endline}}
\newcommand{\pushline}{\Indp}
\newcommand{\popline}{\Indm\dosemic}
\let\oldnl\nl
\newcommand{\nonl}{\renewcommand{\nl}{\let\nl\oldnl}}
\makeatother

\hypersetup{
    colorlinks,
    citecolor=black,
    filecolor=black,
    linkcolor=black,
    urlcolor=black
}

\lstset{
    language=Python,
    numbers=left,
    numberstyle=\small,
    frame=single,
    tabsize=2,
    breaklines=true,
    basicstyle=\ttfamily\small,
    captionpos=b,
    stringstyle=\color{magenta},
    keywordstyle=\color{blue}\bfseries,
    numberstyle=\color{black}
}
\setlength{\parskip}{0.6em}

\makeatletter
\newcommand{\vast}{\bBigg@{4}}
\newcommand{\Vast}{\bBigg@{5}}
\newcommand{\vastl}{\mathopen\vast}
\newcommand{\vastm}{\mathrel\vast}
\newcommand{\vastr}{\mathclose\vast}
\newcommand{\Vastl}{\mathopen\Vast}
\newcommand{\Vastm}{\mathrel\Vast}
\newcommand{\Vastr}{\mathclose\Vast}
\makeatother

\renewcommand{\footrulewidth}{0.4pt}
\newcommand{\bigCI}{\mathrel{\text{\scalebox{1.07}{$\perp\mkern-10mu\perp$}}}}
\renewcommand{\baselinestretch}{1.2}

% \setcounter{page}{3}
\graphicspath{ {images/} }
\lhead{}
\chead{}
\rhead{}

\makeglossaries
\loadglsentries[\acronymtype]{acronyms}
\loadglsentries{glossary}

% \setlength{\parindent}{1.25cm}
\setlength{\parindent}{0pt}
\setlength{\parskip}{10pt}
\setlength{\columnsep}{0.5125cm}
\renewcommand{\baselinestretch}{1.2}

\fancypagestyle{IHA-fancy-style}{%
  \fancyhf{}% Clear header and footer
  \fancyhead[L]{\textit{This thesis is performed by: Vi Thanh Dat - 20164803 - CNTT2.02 - K61}}
  \fancyfoot[R]{\thepage}% Custom footer
  \renewcommand{\headrulewidth}{0.4pt}% Line at the header visible
  \renewcommand{\footrulewidth}{0pt}% Line at the footer visible
}
% Redefine the plain page style
\fancypagestyle{plain}{%
  \fancyhf{}% Clear header and footer
  \fancyhead[L]{\textit{This thesis is performed by: Vi Thanh Dat - 20164803 - CNTT2.02 - K61}}
  \fancyfoot[R]{\thepage}% Custom footer
  \renewcommand{\headrulewidth}{0.4pt}% Line at the header visible
  \renewcommand{\footrulewidth}{0pt}% Line at the footer visible
}
\pagestyle{IHA-fancy-style}

\begin{document}

\begin{spacing}{1.25}
    \thispagestyle{empty}
    \thisfancypage{\setlength{\fboxrule}{1pt}\doublebox}{}
    \begin{center}
        {\fontsize{17}{20}\selectfont HANOI UNIVERSITY OF SCIENCE AND TECHNOLOGY} \\
        {\fontsize{13}{17}\selectfont SCHOOL OF INFORMATION AND COMMUNICATION TECHNOLOGY} \\ [0.25cm]
        \textbf{---------------*---------------} \\ [1cm]
        \includegraphics[width=0.2\textwidth]{hust.jpeg} \\ [1cm]
        {\fontsize{25}{30}\selectfont \textbf{GRADUATION THESIS}} \\ [0.25cm]
        {\fontsize{14}{17}\selectfont SUBMITTED IN PARTIAL FULFILLMENT \\
        OF THE REQUIREMENTS FOR THE DEGREE OF} \\ [0.5cm]
        {\fontsize{25}{30}\selectfont \textbf{ENGINEER}} \\ [0.5cm]
        {\fontsize{14}{17}\selectfont IN} \\ [0.5cm]
        {\fontsize{22}{26}\selectfont INFORMATION TECHNOLOGY} \\ [0.5cm]
        {\fontsize{15}{15}\selectfont \textbf{\TODO{TITLE}}}
        \\ [2.25cm]
        \begin{tabular}{ l l }
            Author & : \textbf{Vi Thanh Dat} \\
            Class & : CNTT2.02 K61 \\
            Student ID & : 20164803 \\ [0.5cm]
            Supervisor & : \TODO{supervisor}
        \end{tabular} \\ [2.25cm]
        {\fontsize{17}{20}\selectfont HANOI, 5 - 2021}
    \end{center}
\end{spacing}
\pagebreak

\selectlanguage{english}
\fontsize{13pt}{16pt}
\selectfont

\pagenumbering{roman}
\setcounter{page}{1}
% \begin{spacing}{1.0}
%     \chapter*{Requirements for the Thesis}
%     \section*{Student Information}
%     \begin{itemize}
%         \begin{multicols}{2}
%         \item \textbf{Full name:} VI THANH DAT
%         \item \textbf{Class:} CNTT2.02 K61
%         \item \textbf{Tel:} +84336 863 831
%         \item \textbf{Email:} dat.vt164803@sis.hust.edu.vn
%         \item \textbf{Program:} Full-time engineer
%         \end{multicols}
%         \item \textbf{This thesis is conducted at:} Hanoi University of Science and Technology
%         \item \textbf{Duration:} from 22/01/2021 to
%         24/05/2021
%     \end{itemize}
%     \section*{Objective of the Thesis}
%         Apply deep learning and metric learning to speaker verification in Vietnamese.
%     \section*{Main Tasks of the Thesis}
%         \begin{itemize}
%             \item Study different approaches for speaker verification.
%             \item Study deep learning models and metric learning.
%             \item Apply and improve multi-similarity loss for speaker verification.
%             \item Prepare data, train, and evaluate deep learning models.
%         \end{itemize}
    
%     \section*{The reassurance of student}
%     I, Vi Thanh Dat, guarantee that this thesis is my own efforts under the guidance of \textit{thay tuan anh} and \textit{co trang}. \\
%     All results presented in this thesis are truthful and are not copies of any other works.
%     \begin{minipage}{0.5\textwidth}
%         \hfill
%     \end{minipage}
%     \begin{minipage}[t]{0.5\textwidth}
%         \begin{center}
%         \textit{Hanoi, 1st May, 2021\\Author\\[2.5cm]Vi Thanh Dat}
%         \end{center}
%     \end{minipage}
%     \subsection*{Confirmation of supervisors on the fulfillment of requirements and allowance of defense:}
%     \dotfill\\.\dotfill\\.\dotfill\\.\dotfill\\\\
%     \begin{minipage}{0.5\textwidth}
%         \hfill
%     \end{minipage}
%     \begin{minipage}[t]{0.5\textwidth}
%         \begin{center}
%     \textit{Hanoi, 1st May, 2021\\Supervisor\\[2.5cm]Dr. Nguyen Thi Thu Trang}
%         \end{center}
%     \end{minipage}
% \end{spacing}

% \chapter*{Acknowledgments}


\pagebreak

\chapter*{Abstract}
Placeholder

\chapter*{Tóm tắt}
Placeholder

\chapter*{Summary}
The thesis is divided into 5 chapters:
\begin{itemize}
    \item Chapter 1 describes the problem of training neural networks at scale, as well as the object detection problem that would be used as a case study.
    \item Chapter 2 presents the theoretical background as well as related works, including machine learning, object detection and distributed training.
    \item Chapter 3 details the design and implementation of BK.Synapse - the proposed framework - and a case study where it is applied to an object detection problem.
    \item Chapter 4 reports on the results of several experiments and benchmarks.
    \item Chapter 5 presents our conclusions and lays out plans for future works.
\end{itemize}

\pagebreak
\pagenumbering{gobble}
\tableofcontents
\pagebreak
\pagenumbering{arabic}
\listoffigures
\listoftables

% \addcontentsline{toc}{chapter}{List of Abbreviations}
\printglossary[type=\acronymtype,style=long, title=List of Abbreviations]
% \addcontentsline{toc}{chapter}{Glossary}
\printglossary
\pagebreak
\setcounter{page}{1}

\subfile{introduction.tex}
\subfile{preliminaries.tex}
\subfile{propose.tex}
\subfile{results.tex}

\chapter{Conclusions}
.

\pagebreak
\bibliographystyle{plain}
\bibliography{mybib}
\subfile{appendix.tex}
\end{document}

% 2 chương cho Propose: 
% - optimize VoxCeleb
% - SV for low resource languages
% -> ko cần build large language corpus như thế nhưng vẫn

% 1. Introduction
% 2. Background (~10-15 trang)
% 3. Optimize dataset VoxCeleb ()
% - Trình bày baseline
% - Phương pháp 
% - Có experiment luôn

% 4. Proposal low resource language
% - Lấy cái baseline từ chương 3, however 

% Thực nghiệm: 
% - Baseline pretrained trên tiếng Việt
% - train chung tiếng việt với VoxCeleb
% - Trước và sau enhance
% - train từ đầu bằng tiếng Việt
% - Finetune với tiếng việt
% - Finetune với tiếng Việt + loss


% 3 & 4 trình bày như paper: intro, method, experiments (có link đến low resource languages)

% 5. Experiment and evaluation ()